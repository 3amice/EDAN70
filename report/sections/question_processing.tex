\subsection{Preprocessing of questions}
Given a question, the system will convert the question
to a format which can be used to perform a query to the indexed
database. This is done by blacklisting all characters which
are not present in the swedish alphabet or is a number. The question is also
processed through a linear classifier to determine the category of the 
answers.

\subsection{Training the Question Classifier}
For this system, the training data consists of questions 
from the swedish board game Kvitt eller Dubbelt transcribed 
by Juri Pyykk\"o, Rebecka Weegar, Piere Nugues \cite{QASYS}.

The training data consists of 2300 questions containing 9 fields each.
The classifier only uses two fields from each question:
\begin{itemize}
\item Question Formulation - The original question taken from the transcribed
  card.
\item Answer Category - Which was manually determined during transcription. 
  Possible values: concept, location, description, multiplechoice, amount, organization, 
  other, person, abbreviation, verb, title, timepoint, duration, money.
\end{itemize}
A model file was trained using the training data. The model file was then used when the system 
classifies question categories. For example the question \textit{vad heter sveriges huvudstad?} 
%After processing the training set using \texttt{libshorttext}, a model file is created.
%By using this model file, the system can classify 
%what class an answer is expected to have, for example the question \textit{vad heter sveriges huvudstad?} 
yields the following Category-Score pairs:
\begin{center}
  \begin{tabular} {l c}
    \texttt{location}    & \texttt{1.00} \\
    \texttt{concept}     & \texttt{0.69} \\
    \texttt{verb}        & \texttt{0.63} \\
    \texttt{description} & \texttt{0.52} \\
    \texttt{money}       & \texttt{0.50} \\
  \end{tabular}
\end{center}
