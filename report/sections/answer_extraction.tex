After the passage retreival we retrieve a list of paragraphs where we belive the answer exists. 
From these pieces of text our goal is to rank all the words and hopefully the right one in top.

\subsection{Rank nouns}

To find possible answers we decided to focus on nouns and proper nouns. 
Nouns or proper nouns are obtained with the help of a Part of Speech (PoS) tagger, in our case Stagger \cite{stagger}.

From the list of passages we obtain scores for how well every paragraph fits our question, i.e the bm25 score. 
This is used together with the number of occurances of a noun to calculate the rank.
More precisely: For each noun that is found in any of the paragraphs we calculate the rank as follows:

\[ nounrank = \sum_{p\:\in\:paragraphs}bm25(p) \cdot c \]

where bm25(p) is the bm25 score of a paragraph and c is the number of occurances of a word in a paragraph

\subsection{Reranker}

Trained model (liblinear)

features

categories

\subsection{Puncher}

To further improve the ranking we have experimented with a module we call the puncher. It simply boosts 
words with a matching category (stagger) to the predicted categories (libshorttext). 