The purpose of the passage retriever is to reduce the information to a size that is more manageable.
Given a question, the passage retriever translates this into a query, and searches the index for passages 
relevant to this query. Where these passages are sorted by their similarity to the query.
The passage retriver is entirely built upon Lucene Core, an open source, Java-based system, 
that is capable of fast and effective indexing, and smart querying.

\subsection{Wikipedia}
As mentioned before, the Swedish Wikipedia is used as information source, downloaded from Wikimedia {reference here} as wikitext embedded in XML.
To gather the useful text, a python script were used to simply remove everything but the text. 
This script were slightly modified to accomodate both indexing by entire articles, and indexing by paragraphs.

\subsection{Lucene}


\subsubsection{Analyzer}

\subsubsection{Indexing}

\subsubsection{Querying}
